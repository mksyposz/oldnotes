\section{Odróżnanie przestrzeni} 
$\mathbb{Q} \nsim \mathbb{R}$ \\ 
$(0,1) \nsim [0,1]$ \\ 
$[0,1] \nsim [0,1] \times [0,1]$ \\ 
$[0,1] \nsim S^1$ 
\begin{df} 
    $p \in X$ rozspaja $X$ jeżeli $X \setminus \{p\}$ nie jest spójna 
\end{df} 
\subsection{Składowe spójności} 
\begin{df} 
    W przestrzeni topologicznej $X$ składową spójności punktu $x_0 \in X$ nazywamy 
    największy zbiór spójny $C \ni x_0$.
\end{df} 
\begin{uw} \hfill 
    \begin{enumerate}[(1)]
        \item Rozważamy rodzinę $\mathcal A$ wszystkich spójnych $A \ni x_0$. 
        Wtedy $\bigcup \mathcal A$ jest spójna.
        \item Składowe spojności są domknięte w $X$. 
    \end{enumerate}
\end{uw} 
\begin{prz} 
    $Y \nsim I$ 
\end{prz} 
\begin{prz} ~\\ 
    $\{0,1\}^\mathbb N$ bazowe zbiory otwarte są postaci. \\ 
    $\varepsilon = (\varepsilon_1,\ldots,\varepsilon_n)$ \\ 
    $V(\varepsilon) = \{\varepsilon_1\} \times \{\varepsilon_2\} \times \ldots \times 
    \{\varepsilon_n\} \times \{0,1\} \times \{0,1\} \times \ldots$ \\ 
    $\{0,1\}^\mathbb N \setminus V(\varepsilon) = \bigcup\limits_{\tau = 
    (\tau_1,\ldots,\tau_n),\ \tau \neq \varepsilon} V(\tau)$ \\ 
    Składowe spójności w $\{0,1\}^\mathbb N$ są jednopunktowe.
\end{prz} 
\begin{df} 
    Przestrzeń jest $0-$wymiarowa jeżeli ma bazę złożoną ze zbiorów otwarto-domkniętych. 
    \begin{itemize} 
        \item $\operatorname{int}(\emptyset) = -1$ 
        \item przestrzeń $X \neq \emptyset$ jest zerowymiarowa jeżeli ma bazę zbiorów, 
            których brzegi mają wymiar $-1$. 
        \item przestrzeń $X$ ma wymiar $\le 1$, jzęli ma bazę zbiorów, których brzegi mają
            wymiar 0.
    \end{itemize} 
\end{df} 
\begin{tw} Wymiar $[0,1]^n$ wynosi $n$. \end{tw} 
\begin{tw}[Brouwera] 
    Każda funkcja ciągła $f: [0,1]^n \to [0,1]^n$ ma punkt stały. 
\end{tw} 
\begin{uw} 
    Każda ciągła $f: [0,1] \to [0,1]$ ma punkt stały. \\ 
    Stąd $[0,1] \nsim S^1,\ S^1$ nie ma właśnoci punktu stałego.
\end{uw} 
Rozważmy przestrzeń zwartą $K$ oraz $C(K) = \{f: K \to \mathbb R: f \text{ ciągła}\}$ \\ 
$\norm{f}_\infty = \sup\limits_{x \in K} |f(x)| \quad \rho(f,g) = \norm{f-g}_\infty$
\begin{df} \hfill 
    \begin{itemize} 
        \item $P \subseteq C(K)$ jest pierścieniem, jeżeli $P$ jest zamknięta na dodawanie
            i mnożenie.
        \item $P$ rozdziela punkty jeżeli \cancel{to robi}, dla $x,y \in K 
            x \neq y$ istnieje $p \in P,\ p(x) \neq p(y)$
    \end{itemize} 
\end{df} 
\begin{tw}[Stone'a-Weiestrassa]
    Jeżeli $K$ jest przestrzenią zwartą i $P \subseteq C(K)$ jest pierścieniem 
    rozdzielającym punkty $K$ zawierającym funkcję stałe, to $\overline P = C(K)$
\end{tw} 
\begin{wn}[Weiestrass] Wielomiany leżą gęsto w $C[a,b]$. \end{wn} 
\begin{wn} Wielomiany dwóch zmiennych leżą gęsto w $C[0,1]^2$ \end{wn} 
\begin{wn} Dla dowolnych przestrzeni zwartych $K_1$ i $K_2$ funkcje o rozdzielnych
zmiennych leżą gęsto w $C(K_1 \times K_2)$ \end{wn}
\begin{wn} Wielomiany trygonometryczne\footnote{kombinacja liniowa $\sin nx \cos kx$}
    leżą gęsto w $C[0,2\pi]$ \end{wn} 
\begin{lem}[Diniego]
    Jeżeli $f_n \in C[0,1], f_1 \le f_2 \le \ldots$ oraz granica punktowa 
    $f(x) = \lim\limits_{n \to \infty} f_n(x) $ jest ciągła, to $f_n \rightrightarrows f$
    \begin{dd} 
        Nie wprost, załóżmy, że istnieje $\varepsilon > 0 \ \norm{f-f_n} \ge \varepsilon$
        $F_n = \{x \in [0,1]: |f(x)-f_n(x)| \ge \epsilon \} \neq \emptyset$. 
        Wtedy $F_1 \supseteq F_2 \supseteq \ldots$ więc istnieje $x_0 \in \bigcap\limits
        _{n=1}^\infty F_n$ sprzeczność.
    \end{dd} 
\end{lem} 
\begin{lem} 
    Funkcja $s(t) = \sqrt(t)$ jest jedostajną granica wielomianów.
    \begin{dd} 
        $w_1(t) \equiv 0$ \\ 
        $w_{n+1} (t) = w_n (t) + \frac{1}{2} (t - w_n^2 (t))$ sprawdzimy prez indukcję, 
        że $w_1(t) \le w_2(t) \le \ldots \sqrt t$ \\ 
        $g(t) = g(t) + \frac{1}{2} (t - g^2(t)) \to g(t) = \sqrt t$, czyli \\ 
        $w_n \rightrightarrows s$ z lematu Diniego. 
    \end{dd} 
\end{lem} 
Rozważmy pierścien $P \subseteq C(K),\ P$ rozdziela punkty i zawiera stałe. \\ 
$a, b \in K$ \\ 
$a \neq b$ \\ 
Istnieje $h \in P \ h(a) \neq h(b)$. Niech $g(x) = \frac{h(x) - h(a)}{h(b)-h(a)} \in P$. 
$g(a) = 0 \ g(b) = 1$ \\ 
\rule{2cm}{0.4pt} \\ 
Ustalmy $f \in C(K)$. Dla $ a \neq b$ mamy $g \in P \ g(a) = 0,\ g(b)=1$ \\ 
Rozważmy $p_{a,b} (x) = (f(b)-f(a))g(x) + f(a)$ \\ 
$p_{a,b} \in P \ p_{a,b} (a) = f(a), \ p_{a,b} (b) = f(b)$ \\ 
\rule{2cm}{0.4pt}\\ 
$|p|  = \sqrt{p^2}$. Stosując $w_n \rightrightarrows s,\ s(t) = \sqrt t$ \\ 
$|p| = \lim\limits_{n \to \infty} w_n(p^2),\ w_n(p^2) \in P$ \\ 
\rule{2cm}{0.4pt}\\ 
Jeżeli $p_1,p_2 \in P$ to $\max(p_1,p_2), \min(p_1,p_2) \in \overline P$ \\ 
Ogólniej $p_1,\ldots,p_n \in P$, to $\max(p_1,\ldots,p_n) \in \overline P, \ 
\min(p_1,\ldots,p_n) \in \overline P$ \\ 
\rule{2cm}{0.4pt}\\ 
Ustalmy $f \in C(K),\ \varepsilon > 0$ \\ 
$a \neq b \Rightarrow \exists p_{a,b} \in P \ p_{a,b} = f(a),\ p_{a,b} = f(b)$ \\ 
$U_{a,b} = \{x: p_{a,b} (x) < f(x)+\varepsilon \}$ \\ 
$V_{a,b} = \{x: p_{a,b} (x) > f(x)-\varepsilon \}$ \\ 
Ustalmy $b\ \{U_{a,b}: a \in K\}$ podpokrycie $K$ \\ 
$K = \bigcup\limits_{i=1}^n U_{a_i,b},\ p_b = \min\limits_{i \le n} p_{a_i,b} \in 
\overline P$ \\ 
$p_b (x) < f(x) + \varepsilon \ x \in K$ \\
Dla $x \in V_b = \bigcap\limits_{n=1}^n V_{a_i,b} \ p_b (x) > f(x) - \varepsilon$ \\ 
$\{V_b: b \in K\} - $ pokrycie $K$ \\ 
Istnieje $b_1,\ldots,b_k,\ldots$ $p \in \max\limits_{i \le k} p_{b_i} \in \overline P$ \\ 
$\bigcup\limits_{i = 1}^k V_{b_i} = K$ \\ 
$p(x) < f(x) + \varepsilon$ \\ 
$p(x) > f(x) - \varepsilon$ \\ 
$B_\varepsilon (f) \cap \overline P \neq \emptyset$ \\ 
$\overline P = C(K)$ \hfill \qed

